% This document is distributed under the creative commons license:
% Attribution-ShareAlike 4.0 International (CC BY-SA 4.0)
% For the full license, see:
% http://creativecommons.org/licenses/by-sa/4.0/
\documentclass{beamer}

\usepackage{hyperref}
\usepackage{listings}
\usepackage{color}
\usepackage{minted}
\usepackage[normalem]{ulem}
\usepackage{beamerthemeOTS}

\title
{Design Patterns: an Introduction}

%\subtitle
%{Include Only If Paper Has a Subtitle}

\author[B. Bleuz\'e \and H. Schol] % (optional, use only with lots of authors)
{\textsc{Beno\^it Bleuz\'e} \and \textsc{Haiko Schol}}
% - Give the names in the same order as the appear in the paper.
% - Use the \inst{?} command only if the authors have different
%   affiliation.

\date[2014] % (optional, should be abbreviation of conference name)
{27/08/2014}

\subject{Design Patterns}

\begin{document}
\begin{frame}
  \titlepage
\end{frame}

\begin{frame}
\frametitle {Introduction}
  \textbf{Design Patterns}, what is all the fuss about?
\begin{itemize}
 \item Mantra uttered over and over by old bearded gurus
 \item Scary diagrams
 \item Abstract, mysterious names
 \item universal magical answer to the spaghetti code I end up having at the end of my project
\end{itemize}
\centering \only<2->{\alert {Let's demystifies\dots}}
\end{frame}

\begin{frame}
  \frametitle{Outline}
  \tableofcontents
  % You might wish to add the option [pausesections]
\end{frame}

\section{Engineering techniques}
\begin{frame}
\frametitle {History}
\begin{itemize}
 \item \textbf{Design Patterns}: invented by \only<1>{software}\only <2->{\emph{\sout{software}}} Architect, \emph{Christopher Alexander}.\pause
 \item \pause \textbf{Design Patterns: Elements of Reusable Object-Oriented Software} 1994, \\
authored by the \textbf{Gang Of Four} (\emph{Erich Gamma, Richard Helm, Ralph Johnson and John Vlissides}) popularised the concept in Computer science.
 \item \pause Since then, countless books on how to apply many more patterns to many specific languages and problems.
\end{itemize}




\end{frame}
\begin{frame}
 \frametitle{Reusable techniques solving recurring problems}
They are \textbf{not}:
  \begin{itemize}
   \item tools (tools are editors or IDEs)
   \item ready made one fits all solutions
  \end{itemize}
\end{frame}



\begin{frame}
\frametitle {Reusable techniques solving recurring problems}
A Pattern is only complete if it comes with:
\begin{itemize}
 \item a typical situation along with why it causes trouble
 \item a solution and the reason it is a good one.
 \item a context, when to apply to it.
\end{itemize}
\end{frame}

\section{Categories}

\begin{frame}
\frametitle {GoF Object Oriented Categories}
 In the book \textbf{Design Patterns}, the are the following categories:
\begin{description}
 \item [Creational] create resources instead of direct instantiation
 \item [Structural] organise the structure of the data: hierarchy, composition
 \item [Behavioral] drive the data flow, communication between objects
\end{description}
\end{frame}

\begin{frame}
\frametitle {GoF Creational Patterns}

\begin{description}
 \item [Abstract Factory] groups object factories that have a common theme.
 \item [Builder constructs] complex objects by separating construction and representation.
 \item [Factory Method] creates objects without specifying the exact class to create.
 \item [Prototype] creates objects by cloning an existing object.
 \item [Singleton] restricts object creation for a class to only one instance.
\end{description}
\end{frame}

\begin{frame}
\frametitle {GoF Structural Patterns}
\begin{small}
\begin{description}
 \item [Adapter] allows classes with incompatible interfaces to work together by wrapping its own interface around that of an already existing class.
 \item [Bridge] decouples an abstraction from its implementation so that the two can vary independently.
 \item [Composite] composes zero-or-more similar objects so that they can be manipulated as one object.
 \item [Decorator] dynamically adds/overrides behaviour in an existing method of an object.
 \item [Facade] provides a simplified interface to a large body of code.
 \item [Flyweight] reduces the cost of creating and manipulating a large number of similar objects.
 \item [Proxy] provides a placeholder for another object to control access, reduce cost, and reduce complexity.
\end{description}
\end{small}
\end{frame}

\begin{frame}
\frametitle {GoF Behavioral Patterns}

\begin{scriptsize}
\begin{description}
 \item [Chain of responsibility] delegates commands to a chain of processing objects.
 \item [Command] creates objects which encapsulate actions and parameters.
 \item [Interpreter] implements a specialized language.
 \item [Iterator] accesses the elements of an object sequentially without exposing its underlying representation.
 \item [Mediator] allows loose coupling between classes by being the only class that has detailed knowledge of their methods.
 \item [Memento] provides the ability to restore an object to its previous state (undo).
 \item [Observer] is a publish/subscribe pattern which allows a number of observer objects to see an event.
 \item [State] allows an object to alter its behavior when its internal state changes.
 \item [Strategy] allows one of a family of algorithms to be selected on-the-fly at runtime.
 \item [Template method] defines the skeleton of an algorithm as an abstract class, allowing its subclasses to provide concrete behavior.
 \item [Visitor] separates an algorithm from an object structure by moving the hierarchy of methods into one object.
\end{description}  
\end{scriptsize}
\end{frame}

\begin{frame}
\frametitle {Other known categories}

\begin{description}
 \item [Architectural] Large application scale patterns: core/plugin, data stores\dots
 \item [Concurrency] concurrent programming techniques, including locking mechanisms, thread pooling\dots
 \item [Testing] code testing patterns: mock objects, set-up/tear-down\dots
 \item [Domain specific] Some patterns are very much domain specific
\end{description}  
\end{frame}

\section{Concrete examples}

\subsection{MVC Pattern}
\begin{frame}
\frametitle{MVC: Principle}
NICE GRAPH BUT not going further
\end{frame} 

\subsection{Factory Pattern}
\begin{frame}
\frametitle{Factory Pattern: Principle}
NICE GRAPH
\end{frame} 

\begin{frame}[fragile]
\frametitle{Factory Pattern: Code} 
\begin{minted} [fontsize=\footnotesize] {python}
# Beautiful code
\end{minted}
\end{frame}

\subsection{Adapter Pattern}
\begin{frame}
\frametitle{Adapter Pattern: Principle}
NICE GRAPH
\end{frame} 

\begin{frame}[fragile]
\frametitle{Adapter Pattern: Code} 
\begin{minted} [fontsize=\footnotesize] {python}
# Beautiful code
\end{minted}
\end{frame}



\section{WordS of Caution}
% language specific
% always evolving: trends and falls
\begin{frame}
 \frametitle{Limitations}
\begin{itemize}
 \item Language specific patterns
 \item Trends
 \item Technical advances, obsolescence
\end{itemize}
\end{frame}

\begin{frame}
\frametitle{Singleton Pattern: Principle}
\begin{itemize}
 \item One instance shared all over the code: "global object"
 \item Examples: config managers, Factories, Facade objects
\end{itemize}
\end{frame} 

\begin{frame}
\frametitle{Controversies: Singleton Pattern}
\begin{itemize}
 \item very much used, BUT
 \item introducing global state never encouraged
 \item tricky thread safety at initialisation (ultimately flawed in c++ due to uncertainty of dynamic loading order)
 \item overused: situations REALLY needing a singleton are rare
\end{itemize}
\end{frame} 



\begin{frame}[fragile]
\frametitle{Adapter Pattern: Code} 
\begin{minted} [fontsize=\footnotesize] {python}
# Beautiful code
\end{minted}
\end{frame}


\section{Conclusions}
\begin{frame}
\frametitle{Conclusions}


There is so much more we can show...
 
\end{frame}

% \begin{frame}
% \frametitle{Questions?}
% \begin{center}
% %   \includegraphics[height=.5\textheight]{Code-Refactoring-Cat-in-Bathtub.gif}
% % look at animate.
% \end{center}
% \end{frame}

\end{document}
