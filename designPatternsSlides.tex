% This document is distributed under the creative commons license:
% Attribution-ShareAlike 4.0 International (CC BY-SA 4.0)
% For the full license, see:
% http://creativecommons.org/licenses/by-sa/4.0/
\documentclass{beamer}

\usepackage{hyperref}
\usepackage{listings}
\usepackage{color}
\usepackage{minted}
\usepackage[normalem]{ulem}
\usepackage{beamerthemeOTS}

\title
{Design Patterns: an Introduction}

%\subtitle
%{Include Only If Paper Has a Subtitle}

\author[B. Bleuz\'e \and H. Schol] % (optional, use only with lots of authors)
{\textsc{Beno\^it Bleuz\'e} \and \textsc{Haiko Schol}}
% - Give the names in the same order as the appear in the paper.
% - Use the \inst{?} command only if the authors have different
%   affiliation.

\date[2014] % (optional, should be abbreviation of conference name)
{27/08/2014}

\subject{Design Patterns}

\begin{document}
\begin{frame}
  \titlepage
\end{frame}

\begin{frame}
\frametitle {Introduction}
  \textbf{Design Patterns}, what is all the fuss about?
\begin{itemize}
 \item Mantra uttered over and over by old bearded gurus
 \item Scary diagrams
 \item Abstract, mysterious names
 \item universal magical answer to the spaghetti code I end up having at the end of my project
\end{itemize}
\centering \only<2->{\alert {Let's demystifies\dots}}
\end{frame}

\begin{frame}
  \frametitle{Outline}
  \tableofcontents
  % You might wish to add the option [pausesections]
\end{frame}

\section{Engineering techniques}
\begin{frame}
\frametitle {History}
\begin{itemize}
 \item \textbf{Design Patterns}: invented by \only<1>{software}\only <2->{\emph{\sout{software}}} Architect, \emph{Christopher Alexander}.\pause
 \item \pause \textbf{Design Patterns: Elements of Reusable Object-Oriented Software} 1994, \\
authored by the \textbf{Gang Of Four} (\emph{Erich Gamma, Richard Helm, Ralph Johnson and John Vlissides}).
 \item \pause Since then, countless books on how to apply many more patterns to many specific languages and problems.
\end{itemize}




\end{frame}
\begin{frame}
 \frametitle{Reusable techniques solving recurring problems}
They are \textbf{not}:
  \begin{itemize}
   \item tools
   \item ready made one fits all solutions
  \end{itemize}

\end{frame}



\begin{frame}
\frametitle {An analogy: Surgical knots}


\end{frame}
%history, genesis
% GoFs
\section{Categories}

\section{Concrete examples}

\section{Word of Caution}
% language specific
% always evolving: trends and falls

\section{Conclusions}
\begin{frame}
\frametitle{Conclusions} 
There is so much more we can show...
 
\end{frame}

% \begin{frame}
% \frametitle{Questions?}
% \begin{center}
% %   \includegraphics[height=.5\textheight]{Code-Refactoring-Cat-in-Bathtub.gif}
% % look at animate.
% \end{center}
% \end{frame}

\end{document}
